\cleardoubleoddpage
\chapter{Conclusion and Outlook}
\label{sec:conclusion}

\section{Key Findings}
\label{subsec:results_keyfindings}

This thesis set out to investigate sentiment and linguistic patterns in YouTube music video comments across genres, guided by six research questions.  
The main findings can be summarized as follows:


\begin{enumerate}
\item \textbf{Sentiment and engagement (RQ1).}  
Correlations between comment sentiment and likes were generally weak to modest across genres (mostly $|\rho| < 0.2$).  
Some genres (e.g., Nu Jazz, Disco, Latin Rock) showed that more positive or overall positive-toned comments received slightly more likes.  
In contrast, other genres rewarded neutral or even negative comments (e.g., Singer-Songwriter, Folk, Contemporary Classical), suggesting that critique or irony can also attract approval.  
This indicates that audience engagement on YouTube is not universally linked to sentiment but varies by cultural or stylistic context.  
These findings should be interpreted with caution, as many comments received no likes at all, which reduces the stability of correlation-based insights.  
Furthermore, engagement was measured only via likes, capturing just one limited aspect of audience interaction.  

\textit{Expected:} at least a small positive correlation.  

\textit{Interesting:} the direction of correlations varies by genre, with some communities valuing positive reinforcement and others rewarding critical or alternative engagement styles.

\item \textbf{Genre sentiment differences (RQ2).}  
Clear cross-genre differences emerged. Spiritual and smooth genres (e.g., worship, gospel, smooth jazz) showed the most positive average sentiment.  
Aggressive (e.g., death metal, noise, hardcore punk) scored lowest, yet still above zero, indicating that even the “most negative” genres lean slightly positive overall.  

\textit{Expected:} spiritual music positive, aggressive music lower.  

\textit{Interesting:} even the lowest genres remain net-positive.

\item \textbf{Polarization (RQ3).}  
Rap and hip-hop comments often contained both positive and negative elements, producing higher polarization scores.  
By contrast, worship comments were less polarized, dominated by unambiguously positive expressions.  
Between-comment polarization varied systematically across genres, with aggressive and rap-related styles showing the strongest coexistence of enthusiasm and critique.  
Within-comment polarization, while less pronounced, confirmed that individual comments also sometimes mix positive and negative sentiment, further reflecting ambivalence in community communication styles.  

\textit{Expected:} some polarization in rap due to mixed reception.  

\textit{Interesting:} systematic differences in polarization levels between genre groups, with both within- and between-comment measures pointing to cultural contrasts.

\item \textbf{Psycholinguistic features (RQ4).}  
LIWC analysis revealed genre-specific linguistic markers.  
Aggressive genres had the highest frequencies of swearing and conflict terms, while worship and gospel showed high social and religious word use.  
These patterns align with distinct community norms and communication styles.  

\textit{Expected:} more swearing in aggressive genres and more religious vocabulary in worship.

\textit{Interesting:} how consistently these markers separated genre clusters.

\item \textbf{Emoji usage (RQ5).}  
Comments with 1--5 emojis were on average much more positive than comments with none or with excessive emojis ($>$5).  
Moderate emoji use seems to optimize positive expression, while excessive use diminishes this effect.  

\textit{Expected:} emojis linked to positivity.  

\textit{Interesting:} the clear threshold effect with diminishing returns beyond five emojis.

\item \textbf{Popularity and sentiment (RQ6).}  
More popular videos tend to attract slightly less positive comments.  
Here too, engagement was only operationalized through view counts, which do not capture other relevant aspects such as replies or shares.  

\textit{Expected:} more views → more positivity.  

\textit{Interesting:} the opposite trend, possibly reflecting that viral or mainstream content invites more critical or mixed reactions.

\end{enumerate}

Taken together, the results confirm that genre context matters greatly for interpreting online sentiment, and that lexicon-based methods should be applied with caution when analyzing diverse cultural communities.  

\section{Limitations}

While the analyses provide novel insights, several limitations must be acknowledged:

\begin{itemize}
    \item \textbf{Language filtering:} Only english comments were retained, excluding large non-English communities and possibly biasing results for globally popular genres.
    \item \textbf{Tool limitations:} VADER and LIWC cannot fully capture irony, sarcasm, or the positive use of swearing and slang. This limitation could be participially relevant for hip-hop, which may help explain the observed anomaly.
    \item \textbf{Temporal scope:} Restricting to the 100 most recent comments per video reflects current mood but not historical sentiment trends.
    \item \textbf{Genre coverage:} Genres with fewer than 500 comments were excluded, omitting smaller or emerging communities.
    \item \textbf{Engagement measures:} Likes and views are shaped by YouTube’s algorithms and trends, and may not directly represent audience sentiment.
\end{itemize}

\section{Outlook}

Future research can build on this work in several ways:

\begin{itemize}
    \item \textbf{Multilingual extension:} Including non-English comments would allow analysis of global communities and cross-cultural comparisons.
    \item \textbf{Advanced sentiment models:} Transformer-based or fine-tuned large language models could better capture slang, sarcasm, and cultural nuance than lexicon-based tools.
    \item \textbf{Temporal dynamics:} Tracking comment sentiment over time could reveal how community mood shifts around events like album releases or controversies.
    \item \textbf{Broader engagement measures:} Going beyond likes/views to include replies, shares, or watch time would provide a more complete picture of audience interaction.
\end{itemize}

Overall, this thesis provides the first large-scale, genre-aware analysis of sentiment and linguistic markers in YouTube music video comments.  
It demonstrates both expected and surprising patterns, and lays the groundwork for more culturally and methodologically robust future research.
