%%%%%%%%%%%%%%%%%%%%%%%%%%%%%%%%%%%%%%%%%%%%%%%%%%%%%%%%%%%%%%%%%%%%%%%%%%%%%%%%
%% 
%% Methods / Analysis Approach
%% 

\cleardoubleoddpage%  Make sure to start on a new odd page

\chapter{Methods}
\label{sec:methods}
This chapter explains the analysis design and statistical methods, describing what was computed and why. The actual results and figures are presented later in the Results chapter.

\section{Sentiment Scoring (VADER)}
\label{subsec:vader}

We quantify comment-level sentiment using VADER (Valence Aware Dictionary and sEntiment Reasoner) \cite{hutto2014vader}. 
VADER is a lexicon- and rule-based sentiment analysis tool specifically designed for short, informal, 
and emoji-rich social media text. Its foundation is a manually created lexicon of words, phrases, 
slang expressions, and emojis, each of which was assigned a valence score on a scale from $-4$ 
(most negative) to $+4$ (most positive) based on ratings from human annotators. 
This lexicon was validated to capture sentiment in social media communication.

For each comment, VADER returns four sentiment indicators:

\begin{itemize}
    \item \texttt{positive} $\in [0,1]$: proportion of text that conveys positive valence,
    \item \texttt{negative} $\in [0,1]$: proportion of text that conveys negative valence,
    \item \texttt{neutral} $\in [0,1]$: proportion of text that conveys neutral valence,
    \item \texttt{compound} $\in [-1,1]$: a single summary score representing the overall sentiment score.
\end{itemize}

By definition, the three proportional scores sum to one:
\[
\texttt{positive} + \texttt{negative} + \texttt{neutral} = 1.
\]

The \texttt{compound} score is calculated by summing the valence scores of all words and emojis in the text, 
then adjusting them according to syntactic and semantic heuristics (such as negation, degree modifiers, 
punctuation, capitalization, and contrastive conjunctions). This raw sum is normalized using the following formula \cite{hutto2014vader}:
\[
\texttt{compound} = \frac{x}{\sqrt{x^2 + \alpha}},
\]
where $x$ is the unnormalized sum of valence scores and $\alpha = 15$ is a normalization constant. 
This normalization ensures the final \texttt{compound} score remains bounded in $[-1,1]$, with 
values closer to $-1$ indicating strong negativity, $0$ indicating neutrality, and $+1$ indicating strong positivity.

Importantly, VADER’s lexicon also explicitly covers emojis and emoticons. 
These symbols are assigned valence values according to their typical affective meaning. 
Heuristic rules further take into account emoji placement, repetition, and punctuation, 
such that multiple positive emojis or exclamation marks amplify overall positivity, 
while repeated negative symbols strengthen negativity.

\section{Psycholinguistic Features (LIWC)}
\label{subsec:liwc}

We extract psycholinguistic features using LIWC. We focus on families that are
interpretable for community style and affect, including (but not limited to):
\textit{Affect}, \textit{Positive Emotion}, \textit{Negative Emotion},
\textit{Tone}, \textit{Social}, \textit{Swear}, \textit{Anger}, \textit{Anxiety},
\textit{Clout}, and \textit{Authentic}.

\paragraph{Normalization.} LIWC outputs are proportions (\%), which we treat as rates in $[0,1]$
after dividing by 100 for analysis. For visualization we may display percentages;
for statistical modeling we use the $[0,1]$ rates.

\section{Engagement Metrics}
\label{subsec:engagement}
\paragraph{Engagement–sentiment analysis.}

We measure how comment sentiment relates to engagement (likes) in each genre using
\emph{Spearman rank correlations} ($\rho$) between like counts and the four VADER
dimensions (positive, negative, neutral, compound). Spearman’s $\rho$ is less sensitive
to outliers and skewed data than Pearson’s correlation. For genres with at least
$n \geq 500$ comments, we report $\rho$ together with 95\% confidence intervals.

\section{Polarization Metrics}
\label{subsec:polarization}

We measure polarization within comments and across comments to capture mixed sentiment at different levels.

\subsection{Within-Comment Polarization}
\label{subsubsec:within}

Let $\mathrm{pos}$ and $\mathrm{neg}$ denote a comment's VADER positive and negative proportions
(with $\mathrm{pos}+\mathrm{neg}+\mathrm{neu}=1$). We define:
\begin{equation}
\label{eq:polarity_comment}
\mathrm{polarity}_{\mathrm{comment}} \;=\; 2 \times \min(\mathrm{pos},\, \mathrm{neg}) \;\in\; [0,1].
\end{equation}
This score approaches~1 when a comment simultaneously contains strong positive and negative
components (high ambivalence) and is~0 when sentiment is one-sided.

\subsection{Between-Comment Polarization}
\label{subsubsec:between}

For between-comment polarization, we use proportions \emph{aggregated across all comments in a genre}. 
Here, $\overline{\mathrm{pos}}$ denotes the average VADER positive proportion and 
$\overline{\mathrm{neg}}$ the average VADER negative proportion across all comments in that genre. 
We define:
\begin{equation}
\label{eq:polarity_genre}
\mathrm{polarity}_{\mathrm{genre}} \;=\; 2 \times \min(\overline{\mathrm{pos}},\, \overline{\mathrm{neg}}) \;\in\; [0,1].
\end{equation}
High values indicate a genre-level communication style that contains substantial amounts of both 
positive and negative sentiment across comments.

\section{Emoji-Usage Analysis Design}
\label{subsec:emoji_design}
We analyze how emoji frequency relates to sentiment. To optimize interpretability and highlight the range with the highest compound scores, we predefined three mutually exclusive groups at the comment level:
\begin{enumerate}
    \item \textbf{No-emoji}: comments with $0$ emojis.
    \item \textbf{Moderate-emoji}: comments with $1$--$5$ emojis.
    \item \textbf{High-emoji}: comments with $>5$ emojis.
\end{enumerate}


\paragraph{Preprocessing Assumptions.} All analyses are restricted to english
comments, as required by VADER/LIWC, and are performed on the most recent $k=100$
comments per selected video as per the dataset construction. We discuss implications of
these choices in Limitations.

%%%%%%%%%%%%%%%%%%%%%%%%%%%%%%%%%%%%%%%%%%%%%%%%%%%%%%%%%%%%%%%%%%%%%%%%%%%%%%%%
