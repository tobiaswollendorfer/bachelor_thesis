%%%%%%%%%%%%%%%%%%%%%%%%%%%%%%%%%%%%%%%%%%%%%%%%%%%%%%%%%%%%%%%%%%%%%%%%%%%%%%%%
%% 
%% Abstract of your thesis in default document language
%% 

\begin{abstract}
Music is closely linked to emotional experience and the emergence of social communities, and in the digital age, platforms such as YouTube have become central spaces for music consumption and discussion.
YouTube comments in particular provide a large-scale record of spontaneous audience reactions, offering valuable insights into how listeners across different musical genres express themselves online.  

This thesis investigates sentiment and linguistic patterns in YouTube comments on music videos, with a focus on differences between genres. A representative dataset was constructed by combining the JKU dataset of YouTube music links with the Music4All-Onion dataset for genre classification. Using a conflict-resolution algorithm, 50 representative videos were selected for each of 260 genres, yielding over 11,000 videos. From these, 100 recent comments per video were collected via the YouTube Data API, resulting in a dataset of 356,973 english comments across 233 genres. Each comment was preprocessed and analyzed using VADER for sentiment scoring and LIWC for psycholinguistic feature extraction.  

The results highlight clear genre-specific differences in sentiment. 
Spiritual and smooth genres consistently received the most positive comments, 
while hip-hop and aggressive genres showed the lowest sentiment scores despite high engagement. 
Emoji usage was strongly associated with positive sentiment, though excessive use reduced this effect.

\end{abstract}

%% 
%%%%%%%%%%%%%%%%%%%%%%%%%%%%%%%%%%%%%%%%%%%%%%%%%%%%%%%%%%%%%%%%%%%%%%%%%%%%%%%%
