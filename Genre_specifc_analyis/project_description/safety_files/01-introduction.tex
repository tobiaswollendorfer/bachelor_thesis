\cleardoubleoddpage
\chapter{Introduction}
\label{sec:introduction}

\section{Motivation}
\label{sec:introduction:motivation}

Music is one of the most powerful and universal forms of human expression. It shapes cultural identity, regulates emotions, and forms social connection across communities. In the digital era, music consumption has shifted to online platforms, with YouTube emerging as the largest global centre for music streaming and community discussions. With more than 1.2 billion monthly users, YouTube not only gives people access to music but also give them the chance to present some feedback in form of comments, likes, and shares. 

These comments represent a unique opportunity to study audience engagement and genre-specific communication styles. Unlike professional reviews, they capture spontaneous, authentic reactions from listeners across diverse genres and backgrounds. Understanding how these reactions differ across genres provides insight into how music communities construct meaning, express emotions, and build social connections.  

However, analyzing such data is challenging. Social media comments are highly informal, often including slang, emojis, or irony, which complicates interpretation. Traditional sentiment analysis methods—such as lexicon-based models—may fail to capture the nuances of genre-specific language, particularly in culturally distinct communities such as hip-hop. This thesis addresses these challenges by combining computational sentiment analysis with psycholinguistic feature extraction, aiming to reveal how music genres differ in their online communication.
\section{Objectives and Research Questions}

The main objective of this thesis is to investigate how sentiment and linguistic patterns vary across YouTube music video comments, and how these patterns relate to engagement and genre.  
Based on this objective, the following research questions are addressed:

\begin{enumerate}
    \item \textbf{Sentiment and Engagement:}  
    How is the sentiment of YouTube music video comments related to audience engagement, such as the number of likes the comments receive?

    \item \textbf{Sentiment Differences Across Genres:}  
    How do average sentiment scores differ across musical genres, and which genres show the most positive or negative patterns?

    \item \textbf{Polarization:}  
    How polarized are comments within individual messages and across entire genres, and what does this reveal about community communication styles?

    \item \textbf{Psycholinguistic Features:}  
    Which psycholinguistic features (e.g., swearing, social words, conflict terms) distinguish genres, and how do these features correlate with sentiment?

    \item \textbf{Emoji Usage:}  
    How is emoji usage associated with the sentiment of comments, and is there an optimal range of emoji use linked to more positive expressions?

    \item \textbf{Popularity and Sentiment:}  
    Do more popular music videos (measured by view counts) tend to attract more positive or negative comments?

\end{enumerate}

These questions guide the dataset construction, analysis design, and interpretation of results presented in the following chapters.
