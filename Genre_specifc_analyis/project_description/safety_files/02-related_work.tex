\cleardoubleoddpage
\chapter{Related Work}

\subsection*{Social Media Comments Analysis}

The analysis of social media comments has become an increasingly important research field. Most existing studies have focused on platforms such as Twitter and Meta-platforms, including Facebook and Instagram. With more than 1.2 billion monthly users, YouTube is currently the most visited social media platform in the world and the second most visited website overall after Google Search~\cite{statista2024youtubeusers}.

Research on YouTube comment analysis has primarily focused on sentiment analysis, spam detection, and user engagement patterns. Recent studies have proposed advanced methods for filtering spam content from comments~\cite{singha2024spam}. 

Another growing research field is toxicity and hate-speech detection. Hartvigsen et al.\ introduced TOXIGEN, a large-scale dataset of machine-generated and human-written statements designed to capture both explicit and implicit toxic language. Their results showed that models fine-tuned on TOXIGEN outperform baselines in detecting subtle forms of hate speech~\cite{hartvigsen2022toxigen}.
\subsection*{VADER-based Sentiment Analysis}

Many researchers use VADER for analyzing short, informal, and emoji-rich social media text because it handles intensifiers, negation, punctuation and emoticons directly in its rule set~\cite{hutto2014vader}. Beyond the original Twitter-style validation, VADER has been widely applied to YouTube comments to quantify audience polarity. For instance, Chalkias \textit{et al.} (2023) examined 167,987 comments from educational YouTube channels using both VADER and TextBlob, finding that neutral sentiment dominates and that VADER tends to report more neutral statements compared to TextBlob~\cite{chalkias2023exploring}. Another recent study by Zhang (2025) compared VADER and TextBlob on approximately 18,000 YouTube video comments and correlated sentiment with likes and views, showing that while both tools provide useful signals, TextBlob produced more stable correlations in this context~\cite{zhang2025youtube}. In these settings, the compound score serves as a practical and interpretable score for overall valence, enabling large-scale comparisons between topics and channels.
\subsection*{LIWC-based Linguistic Profiling}

In addition to valence detection, LIWC provides psychologically grounded categories spanning affect, social processes, cognitive mechanisms, authenticity, and topical concerns. The approach has been extensively validated and used to profile communication styles and psychological correlates in natural language across online platforms~\cite{tausczik2010psychological,pennebaker2015liwc}. 

In the context of YouTube, LIWC has been used to uncover community norms, such as social bonding, swearing/hostility markers, or religiosity, that are not captured by sentiment alone, thus offering a richer view of how users express themselves. For example, Chae *et al.* (2024) analyzed COVID-19 video transcripts and comments from medical YouTubers using LIWC categories including analytical thinking and emotion (anxiety, anger, sadness), finding that these linguistic/emotional dimensions are associated with viewer engagement and emotional alignment between channel creators and audiences~\cite{chae2024being}.

\subsection*{Why Combine VADER and LIWC?}

VADER and LIWC answer complementary questions. VADER provides a robust social media-tuned estimate of the valence of comments (positive / negative / neutral) with a single comparable compound score suitable for large-scale analyses~\cite{hutto2014vader}. LIWC, decomposes language into psychologically meaningful dimensions (e.g., \textit{Affect}, \textit{Swear}, \textit{Social}, \textit{Cognition}), allowing one to interpret \emph{how} communities communicate, not just \emph{how positive or negative} they are~\cite{pennebaker2015liwc}.Combining both methods makes it possible to map sentiment hierarchies across genres, 
explain those differences through concrete linguistic markers, 
and relate community endorsement (likes) to either overall valence or specific stylistic and psycholinguistic features.


\subsection*{Music-related Social Media Research}

Music-related videos on YouTube generate distinct engagement behaviors compared to other types of content. Research in music psychology has shown that music listening is driven by specific psychological functions such as mood regulation, identity expression, and social connection. These motivations strongly shape how listeners engage with and respond to music content.
Listeners actively select music according to explicit listening intentions and seek playlists aligned with their goals, as shown by the ExIM study \cite{hausberger2025exim}.
Similarly, Arif et al. (2024) conducted a content analysis of BTS music video comments and found strong evidence of parasocial interactions, with fans frequently expressing authenticity, affection, and social bonding with the artist, highlighting the social dimension of the music comment culture~\cite{arif2024parasocial}.

Extending these insights to YouTube more broadly, comments on music videos often contain explicit emotional expressions and personal accounts of why users listen to specific songs. This suggests that the psychological drivers of music listening identified in controlled studies can also be observed directly in user-generated content.

Bauer and Schedl (2019) further contributed to understanding genre-based behavior by introducing the concept of global and country-specific \textit{mainstreaminess}. Using large-scale Last.fm data, they defined measures to capture how closely individual user preferences align with global or local popularity trends. Their results revealed strong cross-country variation, with some regions showing alignment with international mainstream genres, while others emphasized localized and niche listening patterns. This demonstrates that genre preferences are not only individual but also shaped by cultural and community-level contexts~\cite{bauer2019mainstreaminess}.

\subsection*{Summary}

 These studies show that music-related comments on YouTube provide more than casual reactions. They reflect emotional transmission, parasocial interaction, and culturally shaped genre preferences. Building on these insights and leveraging the complementary strengths of VADER and LIWC, this thesis analyzes YouTube music video comments to examine how sentiment and linguistic patterns vary across musical genres and how different communities express themselves within these online spaces.